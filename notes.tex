\documentclass[letter,11pt]{article}

\usepackage[margin=0.75in]{geometry}
\usepackage{amsmath}
\usepackage{sansmathfonts}
\usepackage{listings}
\usepackage[colorinlistoftodos,textwidth=20mm,shadow]{todonotes}

\renewcommand{\familydefault}{lmss}
% Use \qst command for questions that need answering
\newcommand{\qst}[1]{\todo[color=magenta]{#1}}
% Use \lookup for items that need further research
\newcommand{\lookup}[1]{\todo[color=green]{#1}}
% Use \tsk for tasks that still need to be completed
\newcommand{\tsk}[1]{\todo[inline,color=yellow]{#1}}

\newcommand{\regex}[1]{\textit{\small{\textbackslash}\!{#1}}}

% Use ``\cite{NEEDED}'' to get Wikipedia-style ``citation needed'' in document
\usepackage{ifthen}
\let\oldcite=\cite
\renewcommand\cite[1]{\ifthenelse{\equal{#1}{NEEDED}}{\textsuperscript{\texttt{\color{red}[citation~needed]}}\todo{citation needed}}{\oldcite{#1}}}

\author{Eric Wait}
\title{Notes for LLS reconstruction code}
\date{\today}

\begin{document}
% \maketitle
% \listoftodos
% \newpage

\section*{Microscope Output}
	\begin{itemize}
		\item A directory is created sometimes with the format:\\
			\texttt{TimeLapse\regex{d}}

		\item For each stack there is a settings text document.
			\begin{itemize}
				\item This document starts with a user settable string (\texttt{\regex{w+}\_}).
				\item The file name ends with (\texttt{Settings.txt})
			\end{itemize}
			An example of two file names:

			\footnotesize\texttt{642\_50mW\_488\_30mW\_Iter\_000\_Settings.txt}

			\footnotesize\texttt{cell04\_Settings.txt}

		\item Data of interest within this file:
			\begin{itemize}
				\item The start date time group (\texttt{Date : \regex{d+}/\regex{d+}/\regex{d+} \regex{d+}:\regex{d+}:\regex{d+} \regex{['AM','PM']}})
				\item The z interval is in the second token (\texttt{S PZT Offset, Interval (um), \# of Pixels for Excitation (\regex{d}) :	\regex{d+}	\regex{d+}.\regex{d+}	\regex{d+}})
				\item For each channel there will be an entry for the laser where the first token is the channel number and the second token is the excitation wavelength (\texttt{Excitation Filter, Laser, Power (\%), Exp(ms) (\regex{d}) :	N/A	\regex{d+}	\regex{d+}	\regex{d+}})
			\end{itemize}

		\item A 3D tif image is created for each channel and frame.
			The file name contains the following in order:
			\begin{enumerate}
				\item User settable ( \texttt{\regex{w+}\_} ).
				\item Optional iteration number (\texttt{Iter\_\regex{d+}\_}).
				\item Camera number which is only present when there are multiple cameras (\texttt{Cam\regex{D}})
				\item The channel number (\texttt{ch\regex{d}\_}).
				\item The camera number \textit{unused}, always (\texttt{CAM1\_}).
				\item This is the frame number (\texttt{stack\regex{d+}\_}).
				\item The wavelength of the image's channel (\texttt{\regex{d+}nm\_}).
				\item Delta time from start of stack acquisition (\texttt{\regex{d+}msec\_}).
				\item A CPU relative time field (\texttt{\regex{d+}msec}).
					NOTE: there is no underscore after this field.
				\item Position data when doing 3D tiling (\texttt{Abs\_\regex{d+}x\_\regex{d+}y\_\regex{d+}z\_}).
				\item There is a second frame field at the end (\texttt{\regex{d+}t}).
				\item Extension (\texttt{.tif}).
			\end{enumerate}
			An example of two different file names:

			\scriptsize\texttt{642\_50mW\_488\_30mW\_Iter\_000\_CamA\_ch0\_CAM1\_stack0000\_642nm\_0000000msec\_0009004434msecAbs\_000x\_000y\_000z\_0000t.tif}

			\scriptsize\texttt{cell04\_ch0\_CAM1\_stack0000\_488nm\_0000000msec\_0000266777msecAbs\_000x\_000y\_000z\_0000t.tif}

	\end{itemize}
\end{document}
